\documentclass[a4paper,12pt]{article}
    \usepackage[utf8]{inputenc}
    \usepackage[T1]{fontenc}
    \usepackage[english]{babel}
    \usepackage{graphicx}
    \usepackage{geometry}
    \geometry{a4paper,
                 tmargin = 35mm, 
                 lmargin = 25mm,
                 rmargin = 30mm,
                 bmargin = 30mm}
    \usepackage{mathtools}
    \usepackage{amsmath}
    \usepackage{color}
    \usepackage{setspace}
    \usepackage{amsmath,amssymb}
    \usepackage{float}
    \usepackage{listings}
    
    \usepackage{indentfirst}
	\usepackage{subfig}
	    
    \renewcommand\thesection{\Roman{section}.}
    \renewcommand\thesubsection{\thesection\arabic{subsection}}
    \renewcommand\thesubsubsection{}
    
\begin{document}

\begin{titlepage}

	\centering
	{\scshape\LARGE ELTE Faculty of Science\par}
	\vspace{3cm}
	{\scshape\Large Magnets via the metropolis algorithm\par}
    \vspace{1cm}
    {\scshape\large The Ising model\par}
    \vspace{1cm}
	{\large\itshape Alex Olar \par}
    \vspace{3cm}
    \vfill
	{\large 2018 \par}

\end{titlepage}

\onehalfspacing

\section{Motivation}

\par Ferromagnets contain finite size domains in which the spins of all atoms
point in the same direction. When an external magnetic field is applied it is
the nature of the material that the domains start to shift and align. The material
becomes magnetic. However, the process is temperature dependent, therefore when
the temperature gets high enough it looses its magnetism. The point where this 
happens is called the Curie point and is material dependent. This phase transition
is an interesting domain to examine.

\section{The Ising model}

\par Assuming we have a chain of spins where only the nearest neighbors
interact with each other then each spin is in the potential:

\begin{equation*}
    V_{i} = -J\textbf{s}_{i}\textbf{s}_{i+1} - g\mu_{B}\textbf{s}_{i}\textbf{B}
\end{equation*}

\par Where \textbf{B} is the externally applied magnetic field and $\mu_{B}$
is the well known \textit{Bohr-magneton}. J is the so called exchange energy
between spins.

\par With the simulations I want to find out the accuracy of the 
derived parameters depending on the number of particles. For a 
system in state $\Psi_{\alpha}$ the energy of the system is expected
to be:

\begin{equation*}
    E_{\Psi_{\alpha}} = \Big<\Psi_{\alpha}\Big|\sum_{i}V_{i}\Big|\Psi_{\alpha}\Big> = -J\sum_{i=1}^{N-1}s_{i}s_{i+1} - B\mu_{b}\sum_{i=1}^{N}s_{i}
\end{equation*}

\par The best method to simulate a magnetic system is the metropolis algorithm.
In the next section I'll talk about that:

\section{The metropolis algorithm}

\par The steps of the algorithm are the following:

\begin{enumerate}
    \item Initial spin configuration of the system.
    \item Generating a trial configuration by randomly picking a particle and flipping its spin
    than calculating its energy.
    \item If $E_{\Psi_{\alpha}} \geq E_{\Psi_{trial}}$ then accept the state
    otherwise accept with relative probability: $P_{rel} = e^{-\frac{\Delta E}{k_{B}T}}$
    \item If $P_{rel} > random_number$ then accept, otherwise accept the same state as the
    forward state (where \textit{random number} is in the range [0, 1] ).
\end{enumerate}

\section{What I am going to do?}

\par I am going to implement a 1D and a 2D Ising-model with arbitrary, hot and cold 
starting configurations and make great visualizations for it by:

\begin{itemize}
    \item writing the metropolis algorithm
    \item calculating the statistical properties of the system, such
    as magnetization, specific heat and compare them with analytic results
    \item extend the model to second nearest neighbors as well
\end{itemize}

\par My goal is to make a tool (Python notebook or enough time provided 
a web application) that is interactive and shows visualizations of the processes.

\par The obstacles are the hard debugging of python code and acquiring the 
best random generator and finding the best visualization tool out there.

\end{document}